\documentclass[12pt,a4paper,oneside,spanish]{book}
\usepackage[T1]{fontenc}
\usepackage[latin9]{inputenc}
\setcounter{secnumdepth}{3}
\setcounter{tocdepth}{3}

\makeatletter
\makeatother
\usepackage{babel}
\addto\shorthandsspanish{\spanishdeactivate{~<>.}}

\begin{document}
	\chapter{Principio de inducci�n}
		\section{Definiciones}
			\subsection{Conjunto Inductivo}
				Una definici�n inductiva de un conjunto $A$ comprende base, inducci�n y clausura:

				\begin{description}
					\item [{base}] conjunto de uno o mas elementos \flqq{}\emph{iniciales}\frqq{} de $A$.
					\item [{inducci�n}] una o mas reglas para construir \flqq{}\emph{nuevos}\frqq{} elementos de $A$ a partir de \flqq{}\emph{viejos}\frqq{} elementos de $A$.
					\item [{clausura}] determinar que $A$ consiste exactamente de los elementos obtenidos a partir de los b�sicos y aplicando las reglas de inducci�n, sin considerar elementos \flqq{}\emph{extra}\frqq{}.
				\end{description}

				La forma de clausurar es pedir que $A$ sea el m�nimo conjunto que satisface las condiciones de base e inducci�n o en forma equivalente, definir a $A$ como la intersecci�n de todos los conjuntos que satisfacen dichas condiciones.

				\paragraph{Definici�n formal}
				Sean $U$ un conjunto que llamaremos universo, $B$ un subconjunto de $U$ que llamaremos base y $K$ un conjunto no vac�o de funciones que llamaremos constructor.
				Diremos que un conjunto $A$ esta definido inductivamente por $B,K,U$ si es el m�nimo conjunto que satisface:
				
				\begin{itemize}
					\item $B\subseteq A$.
					\item Si $f^{\left(n\right)}\in K$ y $a_{1},\ldots,a_{n}\in A$ entonces $f\left(a_{1},\ldots,a_{n}\right)\in A$ .
				\end{itemize}

			\subsection{Secuencia de formaci�n}
				Sean $U,B,K$ como en la definici�n anterior. Una secuencia $a_{1},\ldots,a_{m}$ de elementos de $U$ es una secuencia de formaci�n para $a_{m}$ si $\forall i=1,\ldots,m$ se verifica que:
				
				\begin{itemize}
					\item $a_{i}\in B$ o bien,
					\item $\exists f\in K$ con $ar(f)=n$ y $0<i_{1},\ldots,i_{n}<i$ tales que $f\left(a_{i_{1}},\ldots,a_{i_{n}}\right)=a_{i}$
				\end{itemize}
				
				Notemos que el conjunto $A$ tiene todos los elementos de $U$ que poseen una secuencia de formaci�n.
				Diremos que $B$ y $K$ definen una gram�tica para las cadenas sintacticamente correctas del lenguaje $A$.

		\section{Demostraciones}
			\subsection{Pertenencia}
				Para probar que un elemento pertenece a un conjunto inductivo, debemos dar su secuencia de formaci�n.

				\paragraph{Ejemplo}
				Sea $L$ el m�nimo conjunto que satisface:
				
				\begin{itemize}
					\item $\lambda,0,1\in L$.
					\item $a\in L\land b\in\{0,1\}\Rightarrow bab\in L$
				\end{itemize}
				
				Probaremos que $110111011\in L$. En efecto posee la siguiente secuencia de formaci�n: $1\Rightarrow111\Rightarrow1011101\Rightarrow110111011$.

			\subsection{No pertenencia}
				Para probar que un elemento no pertenece a un conjunto inductivo, podemos:
				
				\begin{itemize}
					\item Mostrar que no existe una secuencia de formaci�n para el elemento.
					\item Mostrar que si se quita al elemento del conjunto se siguen cumpliendo las clausulas.
					\item \emph{Probar cierta propiedad del conjunto que sirva para excluir al elemento}.
				\end{itemize}
				
				Por ejemplo, para probar que $110111010\notin L$ (definido en el apartado anterior) podr�amos demostrar que todas las cadenas de $L$ comienzan y terminan con el mismo caracter.
				Para demostrar este tipo de propiedades podemos valernos del principio de inducci�n primitiva que se detalla a continuaci�n.

			\subsection{Principio de inducci�n primitiva}
			\paragraph{Enunciado}
				Sea $A\subseteq U$ definido inductivamente por la base $B$ y el constructor $K$, si:
				
				\begin{enumerate}
					\item vale $P(x)$ $\forall x\in B$ y si
					\item para cada $f\in K$ resulta: $P\left(a_{1}\right)\land P\left(a_{2}\right)\land\ldots\land P\left(a_{n}\right)\Rightarrow P\left[f\left(a_{1},a_{2},\ldots,a_{n}\right)\right]$
				\end{enumerate}
				
				entonces vale $P(x)$ $\forall x\in A$.

				\paragraph{Demostraci�n}
				Sea $C$ el conjunto de todos los elementos de $A$ que satisfacen una propiedad $P$, queremos probar que $C=A$.
				
				\begin{itemize}
					\item $C\subseteq A$ es trivial por definici�n del conjunto.
					\item Veamos que $C$ satisface las clausulas de la definici�n inductiva de $A$:
					\begin{itemize}
						\item Sea $x\in B$, luego por \emph{(1)} vale $P(x)$ y entonces $x\in C$ por lo que $B\subseteq C$.
						\item Sean $f^{\left(n\right)}\in K$, $a_{1},a_{2},\ldots,a_{n}\in C$ y $f\left(a_{1},a_{2},\ldots,a_{n}\right)=a$ queremos probar que $a\in C$:
						\begin{itemize}
							\item Por definici�n de $C$ valen $P\left(a_{1}\right),P\left(a_{2}\right),\ldots,P\left(a_{n}\right)$.
							\item Por \emph{(2)} vale $P\left(a\right)$.
						\end{itemize}
						Luego por definici�n de $C$ resulta $a\in C$.
					\end{itemize}
					Dado que $A$ es el m�nimo conjunto que cumple las clausulas de su definici�n inductiva concluimos que $A\subseteq C$.
				\end{itemize}
				
				Puesto que $C\subseteq A$ y $A\subseteq C$ entonces debe ser $A=C$.
\end{document}

\documentclass[12pt,a4paper,oneside,spanish]{book}
\usepackage[T1]{fontenc}
\usepackage[latin9]{inputenc}
\setcounter{secnumdepth}{3}
\setcounter{tocdepth}{3}

\setcounter{chapter}{1}
\usepackage{amsbsy}
\usepackage{amssymb}
\makeatletter
\makeatother
\usepackage{babel}
\addto\shorthandsspanish{\spanishdeactivate{~<>.}}

\begin{document}
	\chapter{Cardinalidad}
		\section{Definiciones}
			\subsection{Funci�n inyectiva}
				Decimos que $f:X\to Y$ es \emph{inyectiva} si: $x_{1}\neq x_{2}\Rightarrow f(x_{1})\neq f(x_{2})$ o bien $f(x_{1})=f(x_{2})\Rightarrow x_{1}=x_{2}$.

			\subsection{Funci�n sobreyectiva}
				Decimos que $f:X\to Y$ es \emph{sobreyectiva} si: $\forall y\in Y$ $\exists x\in X/f(x)=y$.

			\subsection{Funci�n biyectiva}
				Decimos que $f:X\to Y$ es \emph{biyectiva} si es inyectiva y sobreyectiva.

			\subsection{Conjuntos equipotentes}
				Dos conjuntos $A$ y $B$ tienen la misma cardinalidad (son \emph{equipotentes}) si existe una funci�n biyectiva de $A$ en $B$ y lo notaremos: $\#A=\#B$, $A\sim B$.

			\subsection{Cardinalidad precedente}
				La cardinalidad de un conjunto $A$ es anterior a la de un conjunto $B$ si existe una funci�n inyectiva $f$ de $A$ en $B$ y lo notaremos $\#A\preceq\#B$.
				Si ademas ninguna de las funciones inyectivas de $A$ en $B$ es sobreyectiva	entonces: $\#A\prec\#B$.

			\subsection{Conjuntos finitos}
				Un conjunto es finito cuando es vac�o o equipotente a $\left\{ 1,2,\ldots,n\right\} $ para alg�n $n\in\mathbb{N}$. En caso contrario se dice infinito.

			\subsection{Conjuntos numerables}
				Diremos que un conjunto $A$ es numerable si es finito, o bien resulta que $A\sim\mathbb{N}$ en cuyo caso se dice que $A$ es infinito numerable.
				Si nada de lo anterior aplica se dice que $A$ no es numerable.

			\subsection{Familia de conjuntos}
				Un conjunto $F$ se dice una familia de conjuntos si sus elementos son conjuntos. Diremos que $F$ es una familia indexada de conjunto indice $I$ (no vac�o) si existe una funci�n con dominio $I$ y recorrido $F$.
				Llamando $S_{\alpha}$ (con $\alpha\in I$) a los elementos de la	familia $F$, podemos entonces decir que $F=\{S_{\alpha}/\alpha\in I\}$.

			\subsection{Conjunto de partes}	
				Dado un conjunto $S$, el conjunto de partes de $S$ denotado por $\mathcal{P}(S)$ es el conjunto de todos los subconjuntos de $S$.

		\section{Teoremas}
			\subsection{Teorema de Cantor-Schroder-Bernstein}
				\paragraph{Enunciado}
				Si $\#A\preceq\#B$ y $\#B\preceq\#A$ entonces $A\sim B$. En otras palabras: si existe una funci�n inyectiva de $A$ en $B$ y otra de	$B$ en $A$ entonces existe una funci�n biyectiva de $A$ a $B$.
				\paragraph{Demostraci�n}
				Consultar \flqq{}Daniel J. Velleman. \emph{How to Prove It}.\frqq{} y \flqq{}Richard Hammack. \emph{Book of Proof}.\frqq{} paginas 322 y 232.
				
\pagebreak{}

			\subsection{Cardinalidad de $\mathbb{N\times\mathbb{N}}$}
				\paragraph{Enunciado}
				El producto cartesiano $\mathbb{N\times\mathbb{N}}$ es infinito numerable.
				\paragraph{Demostraci�n}
				Sea $f:\mathbb{N}\to\mathbb{N}\times\mathbb{N}$ dada por $f\left(n\right)=\left(n,1\right)$. Esta funci�n es trivialmente inyectiva.
				Sea $g:\mathbb{N}\times\mathbb{N}\to\mathbb{N}$ dada por $g\left(a,b\right)=2^{a}3^{b}$.
				El teorema fundamental de la aritm�tica nos permite asegurar que esta funci�n es inyectiva.
				Luego por el teorema de Cantor-Schroder-Bernstein concluimos que $\mathbb{N\sim N\times N}$.
				\paragraph{Observaci�n}
				La funci�n $f:\mathbb{N\times\mathbb{N}}\to\mathbb{N}$ dada por $\mbox{\ensuremath{f\left[\left(i,j\right)\right]=\frac{1}{2}\left(i+j-1\right)\left(i+j-2\right)+i}}$ es una biyeccion:
				
				\begin{center}
					\begin{tabular}{c|cccccc}
						$f\left[\left(i,j\right)\right]$ & $1$ & $2$ & $3$ & $4$ & $5$ & $6$\tabularnewline
						\hline 
						$1$ & $1$ & $2$ & $4$ & $7$ & $11$ & $\swarrow$\tabularnewline
						$2$ & $3$ & $5$ & $8$ & $12$ & $\swarrow$ & \tabularnewline
						$3$ & $6$ & $9$ & $13$ & $\swarrow$ &  & \tabularnewline
						$4$ & $10$ & $14$ & $\swarrow$ &  &  & \tabularnewline
						$5$ & $15$ & $\swarrow$ &  &  &  & \tabularnewline
						$6$ & $\swarrow$ &  &  &  &  & \tabularnewline
					\end{tabular}
				\par\end{center}

			\subsection{Corolario}
				\paragraph{Enunciado}
				$\mathbb{N}^{d}\sim\mathbb{N}$. 
				\paragraph{Demostraci�n}
				Lo demostraremos por inducci�n:
				Para $d=1$ vale trivialmente. Veamos ahora que si $\mbox{\ensuremath{\mathbb{N}^{d}\sim\mathbb{N}\Rightarrow\mathbb{N}^{d+1}\sim\mathbb{N}}}$.
				Escribamos $\mathbb{N}^{d+1}=\mathbb{N}^{d}\times\mathbb{N}$. Como $\mathbb{N}^{d}$ es numerable (por hip�tesis inductiva) podemos listar a sus elementos: $\mathbb{N}^{d}=\{a_{1},a_{2},a_{3},\ldots\}$. 
				Sea $f:\mathbb{N}\times\mathbb{N}\to\mathbb{N}^{d+1}$ dada por $f\left(i,j\right)=\left(a_{i},j\right)$	resulta $\mathbb{N}^{d+1}\sim\mathbb{N}\times\mathbb{N}\sim\mathbb{N}$.

			\subsection{Uni�n numerable de conjuntos numerables}
				\paragraph{Enunciado}
				Sean $S_{\alpha}$ conjuntos numerables (finitos o infinitos) y un conjunto indice $I$ tambi�n numerable (finito o infinito) entonces la uni�n de los elementos de la familia $F=\{S_{\alpha}:\alpha\in I\}$, es decir $S={\displaystyle \bigcup_{\alpha\in I}S_{\alpha}}$ sera	tambi�n numerable.
				\paragraph{Demostraci�n}
				Nos pondremos en el peor caso posible: supondremos que tanto los conjuntos $S_{\alpha}$ como el conjunto indice $I$ son infinito numerables.
				Dado que el conjunto indice $I$ es infinito numerable, sin perder generalidad podemos considerar de aqu� en mas que $I=\mathbb{N}$.
				Luego podemos escribir entonces $F=\{S_{\alpha}:\alpha\in I\}=\{S_{i}:i\in\mathbb{N}\}$.
				Dado que $S_{i}$ es infinito numerable, podemos escribir $\mbox{\ensuremath{S_{i}=\{a_{ij}/j\in\mathbb{N}\}=\{a_{i1},a_{i2},a_{i3},\ldots\}}}$.
				Observemos que podemos organizar los elementos de la uni�n de acuerdo a la siguiente tabla:
				
				\[
					\begin{array}{ccccc}
						a_{11} & a_{12} & a_{13} & a_{14} & \cdots\\
						a_{21} & a_{22} & a_{23} & a_{24} & \cdots\\
						a_{31} & a_{32} & a_{33} & a_{34} & \cdots\\
						a_{41} & a_{42} & a_{43} & a_{44} & \cdots\\
						\vdots & \vdots & \vdots & \vdots & \ddots
					\end{array}
				\]

				Luego la funci�n $f:S\to\mathbb{N}\times\mathbb{N}$ dada por $f\left(a_{ij}\right)=\left(i,j\right)$ es inyectiva y como $\mathbb{N}\times\mathbb{N}\sim\mathbb{N}$ resulta que $S$ es numerable.

			\subsection{Cardinalidad infinita mas peque�a}
				\paragraph{Enunciado}
				Para todo conjunto infinito $A$, resulta: $\#\mathbb{N=}\aleph_{0}\preceq\#A$.
				\paragraph{Demostraci�n}
				Sea $A$ un conjunto infinito:
				
				\begin{itemize}
					\item Como $A$ es infinito resulta $A\neq\emptyset\Rightarrow\exists x_{1}\in A$.
					\item Como $A$ es infinito resulta $A\neq\left\{ x_{1}\right\} \Rightarrow\exists x_{2}\in A/x_{2}\neq x_{1}$.
					\item Como $A$ es infinito resulta $A\neq\left\{ x_{1},x_{2}\right\} \Rightarrow\exists x_{3}\in A/x_{3}\neq x_{1},x_{2}$.
					\item Como $A$ es infinito resulta $A\neq\left\{ x_{1},x_{2},x_{3}\right\} \Rightarrow\exists x_{4}\in A/x_{4}\neq x_{1},x_{2},x_{3}$.
				\end{itemize}
				
				De esta forma se puede construir una sucesi�n $\left(x_{n}\right)_{n\geq1}$ de elementos de $A$ tales que $x_{i}\neq x_{j}$ si $i\neq j$.
				Definimos $f:\mathbb{N}\to A$ dada por $f(i)=x_{i}$ para todo $i\in\mathbb{N}.$
				Como $f$ es inyectiva resulta que $\aleph_{0}\preceq\#A$.
				
\pagebreak{}

			\subsection{Cardinalidad del conjunto de partes}
				\paragraph{Enunciado}
				Para todo conjunto $S$, resulta: $\#S\prec\#\mathcal{P}(S)$.
				\paragraph{Demostraci�n}
				La funci�n $f(x)=\{x\}$ es inyectiva de $S$ en $\mathcal{P}(S)$	por lo que $\#S\preceq\#\mathcal{P}(S)$. Veamos ahora que no existe	funci�n sobreyectiva de $S$ en $\mathcal{P}(S)$.
				Supongamos existe $g:S\to\mathcal{\mathcal{P}}(S)$ sobreyectiva y definamos $\mbox{\ensuremath{B=\{x\in S/x\notin g(x)\}\subseteq S}}$ el conjunto de los elementos de $S$ que no pertenecen a su imagen a trav�s de $g$. Como $g$ es sobreyectiva y $B\in\mathcal{P}(S)$ sabemos que $\exists x\in S/g(x)=B$.
				
				\begin{itemize}
					\item Si $x\in B$: por definici�n de $B$ resulta $x\notin g(x)=B$. Contradicci�n.
					\item Si $x\notin B$: por definici�n de $B$ resulta $x\in g(x)=B$. Contradicci�n.
				\end{itemize}
				
				Por lo tanto $g$ no es sobreyectiva.

			\subsection{Innumerabilidad del continuo}
				\paragraph{Enunciado}
				El conjunto de los n�meros reales no es numerable.
				\paragraph{Demostraci�n}
				Alcanza con probar que el intervalo $(0,1)$ no es numerable pues $(0,1)\sim\mathbb{R}$. En efecto $f(x)=tan\left(x\pi-\frac{\pi}{2}\right)$	o bien $g(x)=ln\left(\frac{1}{x}-1\right)$ demuestran este hecho.
				Representemos los elementos de $(0,1)$ por su expansi�n decimal infinita, por ejemplo $0,229384112598\ldots$. Supongamos que $(0,1)$ es numerable, habr� entonces un primer elemento, segundo, etc. Listemoslos del siguiente modo:
				
				\[
					\begin{array}{cccccc}
						0,\boldsymbol{a_{11}} & a_{12} & a_{13} & a_{14} & a_{15} & \ldots\\
						0,a_{21} & \boldsymbol{a_{22}} & a_{23} & a_{24} & a_{25} & \ldots\\
						0,a_{31} & a_{32} & \boldsymbol{a_{33}} & a_{34} & a_{35} & \ldots\\
						0,a_{41} & a_{42} & a_{43} & \boldsymbol{a_{44}} & a_{45} & \ldots\\
						0,a_{51} & a_{52} & a_{53} & a_{54} & \boldsymbol{a_{55}} & \ldots\\
						 &  & \vdots
					\end{array}
				\]

				Consideremos ahora el numero $b=0,b_{1}b_{2}b_{3}b_{4}b_{5}\ldots$ donde cada d�gito $b_{i}$ puede ser cualquier d�gito excepto $a_{ii}$ (es decir los n�meros en negrita ubicados en la diagonal). Es claro que $b\in(0,1)$ pero es distinto a todos los n�meros del listado ya que difiere de cada numero en por lo menos un d�gito. Esto constituye una contradicci�n, luego el intervalo $(0,1)\sim\mathbb{R}$ no es numerable.

		\section{Ejemplos}
			\subsection{Cardinalidad de $\mathbb{Z}$}
				Puesto que $f:\mathbb{N\to\mathbb{Z}}$ definida por $f(n)=n/2$ (si $n$ es par) y $\mbox{\ensuremath{f(n)=(1-n)/2}}$ (si $n$ es impar) es biyectiva, resulta $\mathbb{Z}\sim\mathbb{N}$.
				En forma alternativa $\mathbb{Z}=\{\ldots,-2,-1\}\cup\{0\}\cup\{1,2,\ldots\}$ es u. n. c. n.
			\subsection{Cadinalidad de $\mathbb{Q}$}
				Podemos escribir a $\mathbb{Q}$ como una u. n. c. n.: $\mathbb{Q}={\displaystyle \bigcup_{k\in\mathbb{N}}A_{k}}$ con $\mbox{\ensuremath{A_{k}=\left\{ \ldots,-\frac{2}{k},-\frac{1}{k},\frac{0}{k},\frac{1}{k},\frac{2}{k},\ldots\right\} }}$.
\end{document}
\documentclass[12pt,a4paper,oneside,spanish]{book}
\usepackage[T1]{fontenc}
\usepackage[latin9]{inputenc}
\setcounter{secnumdepth}{3}
\setcounter{tocdepth}{3}
\usepackage{amsmath}
\usepackage{amssymb}

\setcounter{chapter}{2}
\makeatletter
\makeatother
\usepackage{babel}
\addto\shorthandsspanish{\spanishdeactivate{~<>.}}

\begin{document}
	\chapter{Funciones recursivas primitivas}
		\section{Definiciones}
			\subsection{Aridad}
				Sea $f:A_{1}\times A_{2}\times\ldots\times A_{n}\to B$ llamaremos \emph{aridad} al numero de argumentos que toma la funci�n, es decir $n$ y notaremos $f^{(n)}$.

			\subsection{Funci�n caracter�stica}
				Dado un conjunto $X$, para cada subconjunto $A\subseteq X$ definimos su \emph{funci�n caracter�stica} $\chi_{A}:X\to\left\{ 0,1\right\} $ como:
	
				\[
					\chi_{A}\left(x\right)=
					\begin{cases}
						1 & x\in A\\
						0 & x\notin A
					\end{cases}
				\]

			\subsection{Funci�n num�rica}
				Llamaremos funci�n num�rica a toda funci�n $f:\mathbb{N}^{k}\to\mathbb{N}$ con $k\in\mathbb{N}$. Si $k=0$ identificaremos a dicha funci�n con un numero perteneciente a $\mathbb{N}$.

			\subsection{Funciones base}
				Llamaremos funciones base a las siguientes tres funciones:
				
				\begin{itemize}
					\item La funci�n cero $c^{(n)}:\mathbb{N}^{n}\to\mathbb{N}$ definida por $c^{(n)}\left(X\right)=0$.
					\item Las funciones proyecci�n $p_{k}^{(n)}:\mathbb{N}^{n}\to\mathbb{N}$ definidas por $\mbox{\ensuremath{p_{k}^{(n)}\left(x_{1},x_{2},\ldots,x_{n}\right)=x_{k}}}$.
					\item La funci�n sucesor $s^{\left(1\right)}:\mathbb{N\to\mathbb{N}}$ definida por $s^{\left(1\right)}(x)=x+1$.
				\end{itemize}

			\subsection{Operadores}
				Definiremos dos operadores que nos permitir�n construir nuevas funciones:

				\begin{itemize}
					\item El operador de composici�n $\Phi$ que dada una funci�n num�rica $f^{(n)}$ y $n$ funciones num�ricas de aridad $k$, construye la funci�n num�rica $h$ definida como:
						\begin{align*}
							h:\mathbb{N}^{k}\to & \mathbb{N}\\
							X^{k}\to & h\left(X^{k}\right)=f\left[g_{1}\left(X^{k}\right),g_{2}\left(X^{k}\right),\ldots,g_{n}\left(X^{k}\right)\right]
						\end{align*}
					y que notaremos $h=\Phi\left(f,g_{1},g_{2},\ldots,g_{n}\right)$.
					\item El operador de recursion $R$ que dadas dos funciones num�ricas $g^{(k)}$ y $h^{(k+2)}$ construye una nueva funci�n num�rica $f^{(k+1)}$ definida como
					
						\[
							f\left(y,X^{k}\right)=
							\begin{cases}
								g\left(X^{k}\right) & y=0\\
								h\left[y-1,X^{k},f\left(y-1,X^{k}\right)\right] & y>0
							\end{cases}
						\]
					y notaremos $f=R(g,h)$.
				\end{itemize}

			\subsection{Definici�n inductiva}
				Definimos inductivamente el conjunto de funciones recursivas primitivas ($FRP$) como el menor conjunto tal que:
				
				\begin{itemize}
					\item Las funciones base pertenecen a $FRP$.
					\item Las funciones obtenidas aplicando un numero finito de operaciones de composici�n y recursion sobre elementos de $FRP$ tambi�n pertenecen a $FRP$.
				\end{itemize}

		\section{Ejemplos}
			\subsection{Predecesor natural}
				La funci�n $\widehat{Pd}{}^{(1)}(y)=
				\begin{cases}
					0 & y=0\\
					y-1 & y>0
				\end{cases}$
				es $FRP$ pues:
				
				\begin{enumerate}
					\item $\widehat{Pd}{}^{(1)}(0)=0=c^{(0)}\left(\right)$.
					\item $\widehat{Pd}{}^{(1)}(y)=y-1=p_{1}^{(2)}\left[y-1,\widehat{Pd}{}^{(1)}\left(y-1\right)\right]$.
				\end{enumerate}
				
				por lo que $\widehat{Pd}{}^{(1)}=R\left(c^{(0)},p_{1}^{(2)}\right)$.

			\subsection{Suma}
				La funci�n $\Sigma^{(2)}(y,x)=y+x$ es $FRP$ pues:
				
				\begin{enumerate}
					\item $\Sigma^{(2)}\left(0,x\right)=0+x=x=p_{1}^{(1)}\left(x\right)$.
					\item $\begin{aligned}\Sigma^{(2)}\left(y,x\right) & =y+x=y+x+1-1=\left(y-1\right)+x+1=s^{(1)}\left[\Sigma^{(2)}\left(y-1,x\right)\right]=\\
					& =s^{(1)}\left\{ p_{3}^{(3)}\left[y-1,x,\Sigma^{(2)}\left(y-1,x\right)\right]\right\} =\Phi\left(s^{(1)},p_{3}^{(3)}\right).
					\end{aligned}
				$
				\end{enumerate}
			
				y en consecuencia $\Sigma^{(2)}=R\left[p_{1}^{(1)},\Phi\left(s^{(1)},p_{3}^{(3)}\right)\right]$.

			\subsection{Funci�n potencia}
				Dada una funci�n $f^{(1)}$ definimos $F^{(2)}$ llamada potencia de $f$ como:
				
				\[
					F\left(y,x\right)=
					\begin{cases}
						x & y=0\\
						f\left[F\left(y-1,x\right)\right] & y>0
					\end{cases}
				\]
			 
				y notaremos $F\left(y,x\right)=f^{y}\left(x\right)$.\\
				La funci�n $f^{y}\left(x\right)$ es $FRP$ pues:
				
				\begin{enumerate}
					\item $F^{(2)}\left(0,x\right)=x=p_{1}^{(1)}\left(x\right)$.
					\item $\mbox{\ensuremath{F^{(2)}\left(y,x\right)=f^{(1)}\left[F\left(y-1,x\right)\right]=f^{(1)}\left\{ p_{3}^{(3)}\left[y-1,x,F\left(y-1,x\right)\right]\right\} =\Phi\left(f^{(1)},p_{3}^{(3)}\right)}}$.
				\end{enumerate}
				entonces $F^{(2)}=R\left[p_{1}^{(1)},\Phi\left(f^{(1)},p_{3}^{(3)}\right)\right]$.

		\section{Conjuntos}
			\subsection{Conjunto recursivo primitivo}
				Diremos $A\subseteq\mathbb{N}^{k}$ es un conjunto recursivo primitivo ($CRP$) si su funci�n caracter�stica $\chi_{A}:\mathbb{N}^{k}\to\left\{ 0,1\right\} $ es $FRP$.

			\subsection{Relaciones recursivas primitivas}
				Una relaci�n $R\subseteq\mathbb{N}\times\mathbb{N}$ se dice recursiva primitiva ($RRP$) si es un $CRP$. 
\end{document}